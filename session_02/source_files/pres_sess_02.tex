\documentclass{article}
\usepackage{xcolor}
\usepackage{tikz}
\usepackage{hyperref}
\usepackage{xepersian}
\settextfont{XB Niloofar}
\begin{document}
\textwidthfootnoterule
\title{جلسه‌ی دوم شیوه‌ی ارائه}
\date{}
\maketitle
\section{انواع منابع}
موارد زیر جزء منابعِ تحقیق هستند:
\begin{itemize}
\item کتاب
\item مقاله
\item صفحه‌ی وِب
\end{itemize}
\textcolor{red}{تمرین}: یک نوع منبع دیگر نام ببرد.
\section{فهرست‌نویسیِ منابع} \label{sec:refs}
وقتی کتابی را در فهرست منابع درج می‌کنیم، مواردی مثلِ:
\begin{itemize}
\item نام نویسنده یا نویسندگان
\item عنوان
\item شماره‌ی ویرایش
\item شماره‌ی جلد
\item محلّ نشر
\item ناشر
\item سال نشر
\end{itemize}
می‌توانند ضروری باشند. نحوه‌ی نگارشِ این موارد را، به‌طور دقیق، ناشر تعیین می‌کند\RTLfootnote{منظور از «ناشر»، آن نهادی است، که نوشته‌ی خود را به آن تحویل می‌دهید (تا چاپ کند یا نزد خود نگه دارد).}. مورد زیر، یک نمونه‌ی رایج است:
\begin{center}
\textcolor{gray}{\emph{شاه‌بهرامی، اسدالله، فریدی ماسوله، مرضیه، روش پژوهش و ارائه، ویرایش دوم، تهران: نص، ۱۳۹۵}}
\end{center}
\textcolor{red}{تمرین}: چه مواردی را باید برای مقاله‌های چاپ شده در مجلّه‌های علمی  ذکر کرد؟

\noindent \textcolor{red}{تمرین}: چه مواردی را باید برای مقاله‌های چاپ شده در فراهمایی‌های\RTLfootnote{معادل فارسی فرهنگستان برای «conference»} علمی ذکر کرد؟
\section{مقایسه‌ی انواع منابع از نظر نرخ چاپ}
\vskip 1cm
\begin{tikzpicture}
\draw[->] (0,0) -- (8,0);
\node at (7.5,-0.3) {سرعت};
\node at (6,0.3) {\rl{مقاله (فراهمایی)}};
\node at (3, 0.3) {\rl{مقاله (مجلّه)}};
\node at (0, 0.3) {کتاب};
\end{tikzpicture}


\noindent شاید بتوان این‌طور در نظر گرفت که به‌روزترین مطالب در فراهمایی‌ها مطرح می‌شوند. مجلّات به پژوهش‌های طولانی‌تر پرداخته و کتاب‌ها مطالب اساسی را شرح می‌دهند.


\noindent برخی ناشر‌های کتب علمیِ معتبرِ حوزه‌ی برق و کامپیوتر، \lr{\href{https://www.wiley.com/}{Wiley}}، \lr{\href{https://en.wikipedia.org/wiki/Prentice_Hall}{Prentice Hall}}, \lr{\href{https://www.mheducation.com/}{Mc Graw Hill}}, \lr{\href{https://www.elsevier.com/books-and-journals/morgan-kaufmann}{Morgan Kaufmann}} و \lr{\href{https://www.oreilly.com/}{O'Reilly}} هستند.

\noindent \textcolor{red}{تمرین}: تعدادی از کتاب‌هایی که از این ناشرها، به عنوان مرجع، در درس‌های شما مورد استفاده قرار می‌گیرند را نام ببرید. (کتاب‌ها طبق نمونه‌ی ارائه‌شده در قسمت~\ref{sec:refs} فهرست شوند.)

\noindent \textcolor{red}{تمرین}: چند ناشر معتبر علمی، به همراه اسم مجلّه‌های آن‌ها، در حوزه‌ی برق و کامپیوتر نام ببرید.

\noindent \textcolor{red}{تمرین}: چند فراهمایی معتبر در حوزه‌ی بینایی ماشین (یا حوزه‌ی دلخواه خودتان) معرّفی کنید.

\noindent \textcolor{red}{تمرین}: سنجشِ \lr{SJR}\LTRfootnote{\href{https://www.scimagojr.com/journalrank.php}{https://www.scimagojr.com/journalrank.php}} در مورد این فراهمایی‌ها را تشریح کنید.
\end{document}
