\documentclass{article}
\usepackage{hyperref}
\begin{document}
\title{A Sample Document to Show How LaTeX Works}
\author{Pooryaa Cheraaqee}
\maketitle
\section{Introduction}
This introduces the reader to the problem that we want to solve and its importance.
\section{Related Works} \label{review}
Here we review what has been done in the past for the particular problem we are approaching.

Such data is presented in table~\ref{anghazi}.
% Table of scores
\begin{table}
\caption{The average and deviation of scores for the courses presented in the department}
\label{anghazi}
\begin{tabular}{|l||cc|}
\hline
Course Name& Average Score& Deviation \\
\hline
Programming& 15& 2.5 \\
Operating Systems& 12& 7 \\
\hline

\end{tabular}
\end{table}
% End of scores' table
\section{Fundamentals}
Here the basics are explained.
\section{The Proposed Method}
Here we explain how we've solved the problem. According to our analysis in Section~\ref{review}, we must consider the time complexity.
\end{document}
